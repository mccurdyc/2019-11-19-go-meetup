%!TEX root=main.tex
% mainfile: ../main.tex

\begin{frame}[fragile]
  \frametitle{Patterns}
  \begin{itemize}
    \item No wrong ``solution'', just possibly better ``solutions''
    \pause
    \item ``Bad'' abstractions are worse than no abstractions
    \pause
    \item \textbf{Understand the flow of requests through packages}
    \pause
    \item \textbf{Part of learning is discovering what doesn't work}
  \end{itemize}
\end{frame}

\begin{frame}[fragile]
  \frametitle{Patterns}
  \framesubtitle{Where do I put \dots}

  \begin{itemize}
    \item{Tests}
    \begin{itemize}
      \item \textbf{No \texttt{tests/}}
      \item \texttt{name\_test.go} files remain in the package with the related \texttt{name.go} file
    \end{itemize}
    \item \texttt{cmd/}
      \begin{itemize}
        \item Multiple binaries / ``entrypoints''
      \end{itemize}
    \item \texttt{internal/} vs \texttt{pkg/}
      \begin{itemize}
        \item \texttt{internal/} - \href{https://blog.gopheracademy.com/advent-2016/go-and-package-focused-design/}{``ensures that changes to the API of internal packages will never break an external application''}
      \end{itemize}
    \item Where do I put everything else?
      \begin{itemize}
        \item \texttt{Dockerfile}, \texttt{README.md}, dotfiles, etc.
      \end{itemize}
  \end{itemize}
\end{frame}

{
  \setbeamercolor{background canvas}{bg=solarizedBase02}
  \setbeamercolor{frametitle}{fg=solarizedOrange}
  \color{solarizedBase2}
\begin{frame}[fragile]
  \frametitle{Abstractions}
  \framesubtitle{What are we trying to solve with abstractions?}

  \begin{itemize}
    \item Efficient mental model building
    \item Readability
    \item Reduce maintenance costs
    \item \textbf{Ultimately, speed}
  \end{itemize}

  \vspace{1em}
  Don't abstract just to abstract
\end{frame}
}


\begin{frame}[fragile]
  \frametitle{Patterns}
  \framesubtitle{1. Flat Structure (i.e., ``abstractionless'')}

  \begin{itemize}
    \item This is a great starting place
    \pause
    \item No package abstractions
    \pause
    \item Everything is in \texttt{package main}
    \begin{itemize}
    \item No ``import cycle'' errors
    \end{itemize}
  \end{itemize}
\end{frame}

\begin{frame}[fragile]
  \frametitle{Patterns}
  \framesubtitle{1. Flat Structure (i.e., ``abstractionless'')}

  \begin{verbatim}
    main.go
    server.go
    database.go
    thing1.go # model, view and controller code
    thing1_test.go
    thing2.go # model, view and controller code
    thing2_test.go
  \end{verbatim}
\end{frame}

\begin{frame}[fragile]
  \frametitle{Patterns}
  \framesubtitle{1. Flat Structure (i.e., ``abstractionless'')}

  Challenges:
  \begin{itemize}
    \pause
    \item Mental model construction is difficult from project structure alone
    \begin{itemize}
      \item Ineffective display of ``grouping'', layering and request flow
    \end{itemize}
    \pause
    \item Readability
  \end{itemize}

  \vspace{1em}
  \centering
  \pause
  These become more true as the \textbf{project grows in size}.

\end{frame}

\begin{frame}[fragile]
  \frametitle{Patterns}
  \framesubtitle{1. Flat Structure (i.e., ``abstractionless'')}

  Benefits:
  \begin{itemize}
    \pause
    \item Immediately tackling the problem(s) at hand
    \pause
    \item Gives abstractions time to emerge; if they exist
    \pause
    \item Easy to identify and build abstractions from this point
  \end{itemize}

\end{frame}

\begin{frame}[fragile]
  \frametitle{Patterns}
  \framesubtitle{2. Model-View-Controller (MVC)}

  \begin{verbatim}
    main.go
    pkg/
      handlers/ # package handlers
        thing1.go
        thing2.go
      database/ # package database
        database.go
      models/ # package models
        thing1.go
        thing2.go
      responses/ # package responses
        thing1.go
        thing2.go
  \end{verbatim}
\end{frame}

\begin{frame}[fragile]
  \frametitle{Patterns}
  \framesubtitle{2. Model-View-Controller (MVC)}

  Challenges:
  \begin{itemize}
    \pause
    \item To do well, requires you to use Go \texttt{interfaces}
    \begin{itemize}
      \item If you are new to Go, this could be a challenge
    \end{itemize}
    \pause
    \item Code duplication to avoid circular dependencies
    \begin{itemize}
      \item You will most likely have a model and response for the same type that are tightly-coupled
      \item Controller calls models and builds a view
    \end{itemize}
    \pause
    \item Related ``things'' are ``\textbf{far}''
  \end{itemize}
\end{frame}

\begin{frame}[fragile]
  \frametitle{Patterns}
  \framesubtitle{2. Model-View-Controller (MVC)}

  Benefits:
  \begin{itemize}
    \pause
    \item Centralized logic for interacting with a data store
    \begin{itemize}
      \item Easier to swap technologies (e.g., PostgreSQL to MySQL), if you have abstracted the technology away from the model
    \end{itemize}
    \pause
    \item Standard outside of Go
    \pause
    \item Related ``things'' are ``\textbf{close}''
  \end{itemize}

\end{frame}

\begin{frame}[fragile]
  \frametitle{Patterns}
  \framesubtitle{Addition Patterns (I'm still learning how to apply these)}

  \begin{itemize}
    \item Domain-driven design (DDD)
      \begin{itemize}
        \item Similar goals to micro-services
        \item Separating parts of the business
        \item Domain-specific logic (i.e., for this service, let's do retries)
      \end{itemize}
    \item Hexagonal architecture
  \end{itemize}
\end{frame}


